% This document is designed to produce a french speaking report.

\documentclass{article}
\usepackage[T1]{fontenc}
\usepackage[main=english]{babel}
\usepackage{url}
\usepackage{lastpage}
\usepackage{fancyhdr}
\usepackage{graphicx}
\usepackage[a4paper, margin=2cm, footskip=18.3pt]{geometry}
\usepackage{listings}
\usepackage{pdflscape}
\usepackage{color}

\newcommand{\header} {
    \setlength{\headheight}{30pt}\pagestyle{fancy}
    \fancyhead[L]{\includegraphics[height=20pt]{~/Templates/heig-logo}}\fancyhead[C]{PCO 2023\\ Lab 4}
    \fancyhead[R]{Timothée Van Hove et Aubry Mangold\\\today}\fancyfoot[C]{}
    \fancyfoot[R]{Page \thepage~sur \pageref{LastPage}}\renewcommand{\footrulewidth}{0.3pt}
}

\lstset{frame=tb,
    language=C++,
    aboveskip=3mm,
    belowskip=3mm,
    showstringspaces=false,
    columns=flexible,
    basicstyle={\small\ttfamily},
    numbers=none,
    numberstyle=\tiny\color{gray},
    keywordstyle=\color{blue},
    commentstyle=\color{dkgreen},
    stringstyle=\color{mauve},
    breaklines=true,
    breakatwhitespace=true,
    tabsize=3
}

\begin{document}
    \header


    \section{Introduction}

    The goal of this lab is to implement the management of a shared rail section in a train simulation.
    The source code of the simulator and a program boilerplate for the student program are provided.
    The simulator is a multithreaded program that simulates the behavior of a train on a rail network, leading to
    synchronization problems.
    The student program must implement the management of the shared rail section, the synchronization between the trains
    at the station and an emergency stop system.

    \section{Analysis}
% Talk about the problems arising from having two trains running on two threads that need to be synchoronized
%    Shared resources
%    Critical sections
%    Synchronization
%    Deadlocks

    \section{Conception}
% Talk about how train blocks work in real life
% Talk about which concepts we've used to solve the problem
% Sections: shared section, station, emergency stop


    \section{Implementation}
% Briefly talk about the implementation, giving code examples where necessary


%Code listing
    \begin{lstlisting}
    \end{lstlisting}


    \section{Tests}

    \subsection*{Shared section}
% Describe the tests we've done to ensure the correctness of our implementation. This section is worth some points.
    - Run the provided train route
    - Run a route with the shared section right next to the station
    - Run a route with the shared section right before the station
    - Run a route with the share section separated from the station by a few blocks
    - Run a route with two shared sections
    - Run a route where there is only a station and a shared section

    \subsection*{Station synchronization}

    \subsection*{Emergency stop}

    \section{Conclusion}

    \appendix

    \section{Train routes}
    % images of the train routes used in the tests

\end{document}
